\epigraph{The creation of a thousand forests is in one acorn.}
{\textit{Ralph Waldo Emerson}}

\section{Basics}
\par{
Branches are, in tandem with commits, one of the fundamental nouns in the
\verb+git+ language (though the developers would probably prefer ``framework''
to language). In spite of this, they communicate huge amounts of information
in just a few bytes of data. A branch is essentially a stream of commits with
a link to its genesis and a link to its future. Branches may be spun off from
any commit, and consequently may be spun off from each other with relative
impunity. They cost nearly nothing to make, often little to merge, and only
cause conflicts when people have disagreed on some part of a program-- thus,
they do not create more work than already existed once they are learned.
}

\par{
We will have another thought experiment. Imagine that you are working on a
script which will analyze some data, but there is a huge caveat: you're not
actually sure what each part of the script will do specifically. Even worse,
you won't be able to fill in the blanks in order because you're just not sure
which problems you'll solve first. Without branching, you have a nightmare:
you're going to have to commit various works in progress on top of each other,
with a mess of spaghetti code that will tangle and stall until you find
yourself ready to throw the computer out of the window. It would be nearly
impossible to reconcile the gradual progress as each independent component is
changed independently. However, with branches, we may apply the following
workflow:
}

\begin{enumerate}
    \item Create a skeletal structure
    \item Commit the skeletal structure
    \item Create a branch for each feature to be filled in
    \item As features are tested and validated, merge them into the ``master''
    branch
\end{enumerate}

\par{
We will now consider a more concrete example. Let's say we would like to
create a recipe book with three recipes: one for macaroni and cheese, one for
rice, and one for chocolate chip cookies. We are constrained to only one file
for the recipe, because some automated system will only read one file per
author and therefore we want all of the credit. Following the above workflow,
we can conceive of the following set of steps:
}

\begin{enumerate}
    \item Create the file for the recipes and place headers for each dish
    \item Create a branch off of that version of the file for each recipe
    \item As recipes are completed, merge them into the master branch
\end{enumerate}

\begin{figure}
\begin{centering}
\includegraphics[width=\textwidth]{branching_example.pdf}
\caption{Figure demonstrating the branching workflow for our recipes.
Within-branch commits are blue, branching-off commits are black, and merge
commits are red. Chronological order follows alphabetical order.}
\label{fig:branching_recipe}
\end{centering}
\end{figure}

\par{
From Figure \ref{fig:branching_recipe}, you can see this workflow as an
example. Each recipe has its own branch available, and they are independent.
When they are each merged into master, the recipe file is gradually built.
Note that commits I and J also have I' and J', because the branches build on
top of previous merges as they are merged into master.  
}

\par{
Let's walk through this example. First, create a new file, \verb+recipe.txt+
which has a header for each recipe and whitespace after it. It should look
like this:
}

\begin{verbatim}
Macaroni and Cheese:

Rice:

Chocolate Chip Cookies:

\end{verbatim}

\par{
Commit these headers, and then prepare for git branching. Right now, HEAD is
at the ``tip'' of master (the latest commit to to master is always the
``tip''). Now, to create a new branch, you can run run:
}

\begin{verbatim}
git branch macaroni
\end{verbatim}

Now run

\begin{verbatim}
git branch
\end{verbatim}

\par{
This shows you all branches that exist and which branch you're currently on,
marked with an asterisk and usually highlighted in green. Note that this means
that despite the fact that we've created a new branch, we're not in it yet.
Run
}

\begin{verbatim}
git checkout macaroni
\end{verbatim}

\par{
You can if you re-run \verb+git branch+ that you are now in the
\verb+macaroni+ branch. Now that we are in the macaroni branch, check the git
log. You will notice that now it indicates that you are on master \emph{and}
macaroni. This is because both branches' tip is at the same spot. Add the
following to your recipe file (note that the header is tabbed from the left):
}

\begin{verbatim}
    Ingredients:
        Noodles (preferably corkscrew)
        Cheese
        Butter
        Water
\end{verbatim}

\par{
Commit this list of ingredients and re-run your git log. Now, HEAD points to
macaroni, the branch you're on, and you can see that master is one commit
behind your current branch. Now, we want to add instructions. We're going to
be uselessly vague; copy the following below the recipe:
}

\begin{verbatim}
    Directions:
        1. Boil water
        2. Put noodles in water
        3. Strain water
        4. Mix noodles, cheese, and butter
\end{verbatim}

\par{
Commit the change. Now we're going to think a little bit about the macaroni
recipe while we start working on the rice recipe. We're going to switch to the
master branch, using the same checkout command we did earlier. From there,
create a new branch. There is a handy shortcut to create the branch and
immediately check it out:
}

\begin{verbatim}
git checkout -b rice
\end{verbatim}

\par{
If you check the log, you will see that \verb+rice+ is sharing a tip with
\verb+master+. \verb+macaroni+ is no longer sharing a tip with master,
however. If you examine the files, you will see that your changes have gone
away. No need to fear, however, you can simply checkout between the files to
verify that those changes still exist in \verb+macaroni+. Once you're
satisfied that your very detailed and useful macaroni recipe is recorded, go
ahead and add ingredients for the rice (this is also a useful recipe).
}

\begin{verbatim}
    Ingredients:
        Water
        Rice
\end{verbatim}

\par{
Commit these changes.
}

\par{
Now, we've made changes in two branches apart from master. It would be neat to
visualize them like the figure from earlier. 
I know what you're thinking: now would be a great time for a
GUI, right? The command line can't draw graphs. \emph{Au contraire}, reader. 
Run the following command and prepare to have your mind blown:
}

\begin{verbatim}
git log --all --graph --decorate --oneline
\end{verbatim}

\par{
Similarly to before, you can limit it to a certain number of previous commits.
\verb+all+ just indicates that you would like to see all branches,
\verb+graph+ means what you just saw, and \verb+decorate+ is essential to make
the graph legible; same for \verb+oneline+; entire commit data is really hard
to parse in a graph. It probably wouldn't hurt to make an alias for this in
your git profile.
}

\par{
We can now see that rice's tip is ahead of macaroni's, but that they share a
common ancestor in the master branch, where the tip is. 
Now that we've put our ingredients in
for rice, and we've poked around on the graph, let's get macaroni merged in.
}

\par{
When we go to review our macaroni recipe before merging it, we realize that we
left out some very helpful information! People have no idea how long it will
take to make their macaroni. We'll call it 15 minutes, that's what it takes
me, anyway. So add a commit with the following:
}

\begin{verbatim}
    Cook Time: 15 minutes
\end{verbatim}

\par{
After checking our graph log one more time and noting that now macaroni is
ahead of rice, it's time to merge. Get back to the master branch, then run
}

\begin{verbatim}
git merge macaroni
\end{verbatim}

\par{
\verb+git+ notifies you that it's fast-forwarding, and notes each file and the
changes in them, noting the total number of lines of insertions. There are no
other files and no deletions in this case, but when merging it will indicate
those as well. Check out the graph again and not that now \verb+master+ is
ahead of \verb+rice+. We don't really want to keep around the macaroni branch
since we're done with the macaroni recipe, and it's merged into master. You
can delete the branch via
}

\begin{verbatim}
git branch -d macaroni
\end{verbatim}

\par{
Note that if you try this on the rice branch, \verb+git+ will protect you from
yourself, and ask you if you're sure since it isn't merged into the current
branch.
\verb+git+ is smart enough to know that you would be losing easy access to
your commits.
You could override that behavior with the \verb+-D+ option, but don't do
that-- I would like for you to keep your work so far.
Now let's do some more work on our rice branch, and add some cook times for it.
Add and commit the following:
}

\begin{verbatim}
    Directions:
        1. Add rice and water to rice cooker
        2. Turn on rice cooker
\end{verbatim}

\par{
You will note that git still shows you the last common ancestor of \verb+rice+
and, implicitly, the \verb+macaroni+ branch. You're not sure how long the rice
will take, so you move on to chocolate chip cookies. Uh-oh! You can't check
out master, because that tip now contains contents from the macaroni recipe,
and it's bad practice to introduce unnecessary interdependencies like that. 
Instead, you can actually checkout the last common ancestor of all branches.
Find that on the graph, and check out that commit hash. You will be warned
that you are in a ``detached head'' state. If you make any commits in that
state, they will be lost the second you change branches. That's not good! But
fortunately, there's a simple solution: you can just branch off of that
commit. That's what we came here to do anyway, so branch off of the commit.
}

\par{Similar to before, we're going to make ingredients and directions commits
separately.}

\begin{verbatim}
    Ingredients:
        Sugar
        Flour
        Chocolate Chips
        Water
        Butter
\end{verbatim}

and

\begin{verbatim}
    Directions:
        1. Mix ingredients
        2. Shape dough into 1 oz circles
        3. Bake at 350F
\end{verbatim}

\par{
Commit each set of changes individually, then look at the log as a graph. You
will see that both rice and cookies share a common ancestor, which is correct.
Fantastic! You realize now that rice takes about a half hour, so we can add
the cook time to the rice. Checkout that branch and commit the following:
}

\begin{verbatim}
    Cook Time: 30 minutes
\end{verbatim}

\par{
Review the graph again, noting that now rice is ahead of cookies. As you did
with macaroni, merge the rice branch into master and review the graph. You
will be presented with an editor window asking you to make a commit. This is
because one of your commits in \verb+rice+ is behind a commit that was merged
in from \verb+macaroni+ after their mutual ancestor was committed. Take a
moment to make sure you understand that. If you look at
Figure \ref{fig:branching_recipe}, you can see that commit D of \verb+rice+ is
behind of commit E \verb+macaroni+, and that their last common ancestor was
before any commits in \verb+macaroni+ at commit A. Therefore, git wants you to
make a commit to make it possible to untangle the branches and double-check
that this is what you want to do. The default message, ``Merge branch
\'rice\''', is a fairly sensible thing to accept, so go ahead and merge it.
Just like before, it will note the file change and its insertions. When you
look at the file, you will see that rice recipe was seamlessly integrated into
the file on top of the macaroni recipe. View the graph log again. You will see
that the rice branch's merge also included the macaroni branch merge because
one commit fused both merges into master. This is very good, because it makes
it easy to retrieve prior states of the recipe file. Things as they are now
reflect the commit I' in Figure \ref{fig:branching_recipe}.
}

\par{
Check out the \verb+cookies+ branch, then add and commit the following:
}

\begin{verbatim}
    Bake Time: 10 minutes
\end{verbatim}

\par{
Merge the changes into master. Again, you'll need to make a commit because the
commits in \verb+cookies+ were interspersed with commits in \verb+rice+. Use
the default commit message. Now when you view the graph, you will see that the
cookies branch follows the rice branch due to the commit interspersion, and
the previous branching display is retained between rice and macaroni. Delete
any non-master branches.
}

\par{
Let's say you decide to release the recipes in this state, as a kind of Beta
release. You can actually tag this recent commit, so it's easy to find later,
via
}

\begin{verbatim}
git tag -a v0.1 -m "beta recipe release"
\end{verbatim}

\par{
This will tag the latest commit with an ``annotated'' tag: one that can be
pushed to remotes. This isn't useful to us yet, but it will be in the future.
You can also search for existing tags. We only have one on this repo, so it
will be boring, but you can run
}

\begin{verbatim}
git tag
\end{verbatim}

\par{
Our friendly 0.1 tag will show up. You can get more information on the
specific commit of a tag via
}

\begin{verbatim}
git show tag_name
\end{verbatim}

\par{
If you run that for our tag, you'll see the full commit log for that tag. You
can add the \verb+-s+ flag to give a more abbreviated set of information like
you'd see for a standard log. Tags are often used to indicate releases or
important commits for future reference. When listing tags, you can also do
searches with the * wildcard.
}

\par{
These are the basics. Even with just this knowledge, your ability to manage
changes is drastically increased. They encourage you to create new branches
with the knowledge that its contents will not impact other branches, while
still retaining the ability to easily merge work together. Make branches
frequently to explore new features, bug fixes, or even documentation rewrites.
The cost is almost nothing! Just make sure to periodically clean out stale
branches. 
}

\section{Remotes}
\par{
Remotes are the easiest way to distribute a version control system. Users will
have a local copy, and also a reference copy located on a server somewhere.
Most commonly, it is hosted on GitHub or GitLab. We will use GitHub for these
exercises. Go to the repository you cloned in Chapter 1. Inside that
repository, run the following:
}

\begin{verbatim}
git remote show
\end{verbatim}

\par{
\verb+git+ should report back to you that it has a remote called
\verb+origin+. This is commonly the original repository that you cloned,
hosted somewhere. Cloned repositories should have this set up by default on
GitHub, without your intervention. Let's say you disagree with the way this
repository is set up, or you would like to manage changes under your own
independent profile. You can create what's called a \emph{fork}. This is where
you create a diverging repository on your own profile. It's easy to do on
GitHub. Go to the class repository
\href{https://github.com/jbteves/test_git}{here} and click on ``Fork'' at the
top right. This will create a version of the repo, complete with all history,
on your own personal profile. You can then create and merge your own branches
without impacting the original repository created for this boot camp. 
}

\par{
Once you fork your project, you should see a link for cloning. This link can
also be used to add a remote. Get the link for your fork, and then run
}

\begin{verbatim}
git remote add fork my-fork-url
\end{verbatim}

\par{
This should add a brand new remote. You can see all of your remotes with more
detail by running 
}

\begin{verbatim}
git remote -v
\end{verbatim}

\par{
Now you'll need to use your remotes to keep your branches up to date (this
will effectively work by using merge). Let's say that you would like to make a
new branch and set it to use your fork as an ``upstream:'' that is, where your
commits will go when you push them. Create a branch as normal. HOWEVER, it is
often useful to put your initials as the start of the branch so that other
developers can identify who the branch author is. Therefore, if you make a
branch called \verb+add_addition+, it behooves you to call it 
\verb+JT_add\_addition+
if your initials are JT. If not, please substitute your own initials.
Once you've made this branch, you can make some change and commit it-- it's
quite irrelevant what the change is because your repository is independent of
the class repository. In order to send this change to your remote repository,
run
}

\begin{verbatim}
git push --set-upstream remote-name branch-name
\end{verbatim}

\par{
Now your branch will contain the ability to merge changes from the remote
reference. This is difficult to demonstrate on your own, however. The one
thing you can do is to continue adding commits and pushing them via
}

\begin{verbatim}
git push
\end{verbatim}

\par{
Push a commit or two to the repository, and then do a hard reset to at least
one commit before the last push. Attempt to push. You will get an error
message: ``error: failed to push some refs'' \ldots ``Updates were rejected
because the tip of your current branch is behind its remote counterpart''. 
This error occurs because you have attempt to modify the history of the branch
that you're pushing to. \verb+git+ doesn't like when you do that, so it is
giving you an error. You can correct the error by pulling the changes, and
then attempting to push a new commit. You can pull one of two ways:

\begin{verbatim}
git fetch remote-name branch-name
git merge remote-name/branch-name
\end{verbatim}

\par{
which will give you clear indications of when things are out of date, and give
you standard merge errors, or
}

\begin{verbatim}
git pull
\end{verbatim}

\par{
which is less typing and typically works fine, but can be a bit of a hassle
when there are more conflicts (see next section). Once you've gotten your
local branch back up to date, you can continue making commits. Note that you
could even make commits which reverse the changes you don't like and then push
them; the main idea is to get you back to a point where you have a common
history with the branch. 
}

\par{
It is my duty to inform you that you can in fact bypass the safeguards
described above. However, allow me to warn you now in capital letters that
this action has a very high probability of making you-- you know what's
coming--
A SINNER IN THE HANDS OF AN ANGRY SYSTEM.
Instead of resolving conflicts and making sure that your branch has a state
where it agrees with the remote, you can simply run
}

\begin{verbatim}
git push -f
\end{verbatim}

\par{
This is a force push, and IT IS A BAD IDEA. Why? It will amend history that,
on a shared repository, other people may have branched from, resulting in an
onslaught of merge conflicts as they attempt to reconcile this change in
history. Additionally, it makes it extremely hard to recover the lost commits
without more bizarre maneuvering. In short, don't do it, or you will be a
sinner in the hands of an angry system. To help stress how against force
pushing I am, I will go as far as to say that unless you have a very
compelling reason to do so (sometimes when merging repositories it is
required) I will refuse to help you, for straying from the righteous path
provided for you here and in the next section, which covers conflicts when
merging in more depth.
}

\section{Conflicts}
\par{
Conflicts in \verb+git+ are inevitable. At some point, some programmer will
push a change which conflicts with a change made ``after'' theirs
chronologically, and the difference will need to be reconciled. For an
ultra-simple example, we will return to the \verb+HelloWorld+ repository.
}

\par{
Create a new branch, maybe called \verb+merge_conflict+, based off of the tip
of master. We will assume that while working on the recipe, you discover that
in your rice cooker, it is in fact only 20 minutes to get rice cooked. Go
ahead, make and commit that change in your branch, then check out master. 
Now, let's say that you pull from a remote and there is now a commit where
another maintainer has changed the cook time up to 35 minutes, because they
have a much slower rice cooker (or no rice cooker at all). Because there is no
way to do this automatically, just manually edit the rice cook time directly
on the master branch and commit it. However, consider that this is equivalent
to pulling from a remote repository and seeing the new commit-- this is a
common occurrence. 
}

\par{
For expediency, we will say that you didn't pay close attention when fetching
and merging or pulling, and now you go to merge your own changes into master.
Attempt to merge your branch into master. You get an angry message:
``CONFLICT (content): Merge conflict in recipe.txt. Automatic merge failed;
fix conflicts and then commit the result.''
Unlike interactive programs, a merge conflict will not trigger the editor.
This is because some people prefer to use a more comprehensive IDE to debug
during a merge conflict. Not me, though. I love a good old CLI text editor.
Fire up a text editor and load \verb+recipe.txt+. 
In the center of the file, you see this bizarre addition: 
}

\begin{verbatim}
<<<<<<< HEAD
    Cook time: 35 minutes
=======
    Cook time: 20 minutes
>>>>>>> merge_conflict
\end{verbatim}

\par{
This is telling you that a line or hunk has two versions. It's now up to you
to merge them manually by picking and choosing which pieces survive. The top
one is ``HEAD'', where we are now. The bottom one is ``merge\_conflict'', the
name of the branch we've attempted to merge. Resolve the conflict by turning
the $<<$ and $>>$ enclosed space into a resolution; just list a range of 20-35
minutes instead. When you're done, delete the lines containing HEAD and the
branch name. Stage the change and commit it, noting a merge conflict in the
message. Describe your resolution strategy. When you finish the commit,
\verb+git+ will note that the conflict is resolved. If the commit did not
properly get rid of the notation above, then it will not resolve the conflict.
Add another commit to correct this if it happens.
}

\par{
It's also possible to get this when ``rebasing,'' a history amendment strategy
where a set of commits is played on top of other commits to make it seem like
one continuous line of commits instead of a merge. We could have done this
with the original recipe, for example. For a simple illustration, we will
create another branch, put a few commits on it, and then rebase it onto
master. Go to the recipe and add a few commits to add a recipe for corn on the
cob.
}

\begin{verbatim}
Corn on the Cob:
    Ingredients:
        Corn
        Butter
    Directions:
        1. Place corn on a hot grill until browned
        2. Slather with butter
    Cook time:
        10 minutes
\end{verbatim}

\par{
Commit it in stages like in the first section, then run}

\begin{verbatim}
git rebase master
\end{verbatim}

\par{
You will see that your commits are already up to date. Now create a new
branch, springing from the same common ancestor commit as the corn branch, and
add a S'mores recipe. Immediately merge it into master. 
}

\begin{verbatim}
S'mores:
    Ingredients:
        Graham Crackers
        Marshmallow
        Chocolate
    Directions:
        1. Light marshmallow on fire
        2. Squish marshmallow and chocolate between two crackers
\end{verbatim}

\par{
Now attempt to rebase the corn branch onto master. You will get another
failure, because the s'mores recipe is in conflict with your corn recipe; only
one of them can go directly below chocolate chip cookies. Similar to before,
resolve the conflict in a text editor and make a commit. This time, though,
the bottom message will be your last corn commit and the top will be whatever
the tip of master is. Resolve the conflict as before, and then use \verb+git+
add+ to add the resolution. Rebase will now require you to go through each
commit and resolve the conflict that would be introduced by playing the corn
commits over the top of the s'mores commits. Once you've finished resolving
all conflicts, the rebase will be complete. You can merge the changes into
master. If you look at the git log graph, you will see that the rebased branch
appears to be ``in line'' with the previous commits instead of branched out.
This can be useful if you are trying to condense work from many branches, but
is ultimately much more complex than simply merging. It will also amend commit
history, so branches which others may have branched off of should never be
rebased; it will cause them a huge merge conflict headache. 
}
