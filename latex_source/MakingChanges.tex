\epigraph{They always say time changes things, but you actually have to change
them yourself}{\textit{Andy Warhol}}

\section{Staging Changes}
\par{
It's great to hear about how git can theoretically help manage changes, but we
have yet to actually change anything. This is something of a bummer, so let's
try making some changes. Make sure that you're in the repository you've
initialized (not the one you cloned) and run \verb+ls+. You will note that
there's nothing there at all, because git uses hidden files and folders to
store its data in order to avoid distracting you. It's very polite that way.
Moreover, there's not much interesting in there yet because there are no
changes. Using a command-line editor of your choice, create a file called
\verb+hello.txt+ with ``Hello, world!'' as its only contents.
}

\par{
Right now, your file is part of the working area. In the working area,
everything is exactly as it appears. You can read and write in a text editor
as usual, run scripts, compile code, etc., and whatever is in the working
directory is reflected in those operations. However, before any changes can be
added to the database, they must first go to the \emph{staging area}. This is
essentially where your patches go for you to review and be sure that you want
them before you enter them to the database. In the case of our file, there is
nothing yet staged because we haven't asked git to do so. Our repository is
also empty, so any changes are going to be ``untracked''-- they are neither
part of the database nor even staged to do so. You can always tell the status
of your changes by running
}

\begin{verbatim}
git status
\end{verbatim}

\par{
which say the following:
}

\begin{verbatim}
On branch master

No commits yet

Untracked files:
  (use "git add <file>..." to include in what will be committed)

	hello.txt

nothing added to commit but untracked files present (use "git add" to track)
\end{verbatim}

\par{
So \verb+git+ is telling us what branch we're on (much more on that later),
that we have no commits, and that hello.txt is untracked. Again, this means
that there is no record in any part of the database that this change exists.
The only reason git knows that it exists is because there is a discrepancy
between the database, which records this repository as being empty, and your
working area, which has a file with contents in it. Note that \verb+git+
helpfully tells us the command for staging files: \verb+git add+. Go ahead and
stage your file via
}

\begin{verbatim}
git add hello.txt
\end{verbatim}

\par{
Now run \verb+git status+ again. 
You should see that there is now a ``Changes to be
committed'' field, which has \verb+hello.txt+ listed. 
This is a lot of text for the amount of information we want to know, which is,
``Did my change get staged or not?'' To see an abbreviated version, run 
}

\begin{verbatim}
git status -s
\end{verbatim}

\par{
This should note that \verb+hello.txt+ is added via the ``A'' identifier. For
a complete listing of abbreviated status identifiers, check out Table
\ref{tbl:status}.
}

\begin{table}
\centering
\begin{tabular}{c | c}
\hline
\hline
Abbreviation & Meaning \\
\hline
A & Added \\
M & Modified \\
D & Deleted \\
R & Renamed \\
C & Copied \\
U & Updated; Unmerged \\
'' & Unmodified \\
! & Ignored \\
? & Untracked \\
\hline
\hline
\end{tabular}
\caption{The git status abbreviated meanings. Source: https://git-scm.com/docs/git-status}
\label{tbl:status}
\end{table}

\par{
Before we proceed, it's worth discussing why a staging area is important.
Staging areas are great for making sure that your commits are clean, and that
any changes you made are correctly grouped. There are a variety of tools that
\verb+git+ supplies to aid with this, but our current repository isn't yet
full enough to explore them. For now, it's sufficient to say that a staging
area is like a record of what you think you want to do. Now comes what to some
people is the scary part: \emph{committing}.
}

\section{Commiting Changes}

\par{
Committing changes is not actually all that scary, which I hope will be well
demonstrated in the coming chapters. However, this is a formal entry into the
database, which means that your choice is discoverable. If it's a bad change
that means someone can find it and blame you for it. I think this is a very
negative outlook, however. It is impossible to produce perfect code with every
commit, or even good code, and the facilities exist to undo the commit quite
easily, in fact. Therefore, think of a commit more passively than the word
implies: it is merely a snapshot in time of what you think is the best idea
you have for how things should take shape. It may be wrong, it may be
incomplete, but it is a defined entry you can fall back in if you start moving
in the wrong direction. For this reason, commit early and often. Here are a
few guidelines for making good commits:
}

\begin{enumerate}
\item Leave useful messages, usually in the present passive voice (e.g.,
``Adds a file''
\item Keep related, dependent changes inside of the same commit
\item Keep unrelated changes in separate commits
\end{enumerate}

\par{
Later, we'll take a deeper dive into items two and three, but for now, we'll
focus on ``useful messages.'' Prepare to use a text editor, and run
}

\begin{verbatim}
git commit
\end{verbatim}

\par{
This will fire up your selected text editor. All lines with \# will be
ignored. The first line is supposed to be a summary, limited to 72 characters
ideally. The body below the comment block will be a more detailed message.
Again, individual lines should try to avoid being larger than 72 characters.
In \verb+vim+, you can enforce this with \verb+set textwidth=72+ in your
\verb+.vimrc+. While in your editor, try to leave a useful summary in the top
line. Just make sure it's useful. Here are some examples of useful and
un-useful commit summaries:
}

\begin{enumerate}
    \item ``Makes the git boot camp drill instructor happy'': fails to say
    what the commit actually \emph{does}, which is to add a file that says
    hello.
    \item ``Created file that says hello'': is okay, but does not use the
    present passive voice. 
    \item ``Adds a file in plain text that has the words Hello, world in it in
    a file called hello.txt'': is too verbose, exceeding our 72 character
    limit. It's also needlessly detailed for a summary.
    \item ``Adds a file that says hello'': is good, saying quickly what the
    changes do. 
    \item ``Initial commit: adds a file that says hello'': is good, optionally
    adding the detail that this is the first commit in the repository; some
    people prefer this to be noted, others don't. 
\end{enumerate}

\par{
Save and quit from your text editor, and you will get a message saying that
your commit changed 1 file, and gives the \emph{hash}: an ugly-looking
hexadecimal value. This is the reference that you can use to get to that
snapshot. Don't worry about keeping track of it for now, it's there mostly to
let you know if you want to re-access it soon after the commit. Over time, you
will build a history of commits that you can reference. Each entry in the
history is a pointer to a snapshot in the database, along with the difference
between it and the previous commit, so that \verb+git+ can manage changes
between commits. See Figure \ref{fig:commit_example} for a visual aid.
}

\begin{figure}
    \caption{
        Successive commits and what they refer to.
        Black arrows and alphabetical order show you commit order.
        Blue arrows are drawn from the commit to what they represent in the
        database. White boxes are snapshots and green boxes are differentials.
    }
    \label{fig:commit_example}
    \includegraphics[width=\textwidth]{commit_example}
\end{figure}


\par{
Let's make some more changes to our file. Add the line, ``What a wonderful
world.'' just after the original line. We'd like to see how that changes all
of our snapshots in comparison to before; surprisingly, there's a lot going on
after this small change. Run \verb+git status+ again, and we'll see that this
time \verb+hello.txt+ is modified rather than untracked. That's because
\verb+git+ knows now that we have that file in our database, so any version
which differs is a set of patches on an existing file rather than something to
be ignored. More compactly, we can see that as an ``M'' if we use \verb+git status -s+ again. 
Let's say that at this point, you're distracted by some
paperwork and spend an agonizing ten minutes hashing it out over the phone
with HR. You return to your seat, and see that you've modified something, but
you're just so enraged that your memory of what you did is completely purged.
You'd like to know before you go around staging and committing things. How can
you see that quickly? Run
}

\begin{verbatim}
git diff
\end{verbatim}

\par{
You should see that we have the original file, and a line starting with a plus
before our newest addition. Neato! We decide that this is all we really
wanted, and it's time to stage it, so we do. If we run \verb+git diff+ again,
we'll see nothing. This is because even though the change isn't committed,
it's at least stage, so \verb+git+ figures we already know about that change.
Let's see that we get distracted again. This time, maybe you have to run to a
meeting that should have been an e-mail, just as soon as you had relaxed from
the paperwork fiasco. You return to your desk and re-run \verb+git status -s+
to remind yourself what was hapenning before the meeting. We see that
\verb+hello.txt+ was modified, but now the change is green instead of red (on
most systems; some systems don't support color highlighting). This is because
the change is now staged, and therefore at least partially tracked. We're
frustrated because we don't remember what we changed, and \verb+git diff+
won't tell us. No need to fear; we can see how the staged version compares to
the previous commit via
}

\begin{verbatim}
git diff --cached
\end{verbatim}

\par{
We are reminded yet again of what's hapenned. Ah, peace at last. We got to
stage it when suddenly somebody sends you a message, urging you not to put
exclamation points in your text files because that makes them too exciting for
work. After a heated debate, you find yourself forced to fix this as well.
With a grumble, you edit the file to switch the ! to a . Do this now in your
editor, and do another status check.
}

\par{
Much to your chagrin, you will see that you have a \emph{red} indication that
\verb+hello.txt+ was modified, and a \emph{green} indication as well. This
means that you have staged some changes in hello.txt and not others.
\verb+git+ has kindly alerted you to this automatically. Note that this
highlights something fundamental about git: you are staging \emph{patches},
not \emph{files}. Therefore, once you make changes to the file, there are
actually two versions in play: the version in your working area, and the
version that is staged. We can use \verb+git diff+ to help us tease apart
these different versions. Go ahead and run it with and without
\verb+--cached+. Note that the differences between staged changes and working
area are still tracked: without \verb+--cached+, we see the differences
between the working area and the staging area, and with it, we see the
differences between the previous commit and the staging area. Be careful when
using git diff to understand the differences. 
}

\par{
Now it's time to commit. Before you go and stage that change from . to !, ask
yourself if the commit fits our criteria for a \emph{good} commit. The simple
answer is no: that change is independent of our saying that it's a wonderful
world. Therefore, we should go ahead and commit our addition, then separately
stage and commit the other change. We can use a special flag for \verb+git commit+ 
to just apply a summary message; these are pretty straightforward
changes. You can do this via
}

\begin{verbatim}
git commit -m "Message in quotes here"
\end{verbatim}

\par{
Since we already gave the commit a summary, the text editor won't launch;
\verb+git+ already has all of the information it needs and it commits more or
less instantly. Then we can stage and commit our punctuation change
separately; go ahead and do so. 
}

\par{
We're feeling singsongy today, so we decide to add another line of Louis
Armstrong's masterpiece. Go ahead and add the line ``I think to myself, what a
wonderful world.'' below the others and stage the change. Just as you're about
to commit, however, your nemesis, taker of the exclamation point, announces
that you're limited to one song lyric per file. Ridiculous! But you must
comply. How do you unstage it? To unstage all changes is straightforward: run
}

\begin{verbatim}
git reset -- *
\end{verbatim}

\par{
This tells \verb+git+ to reset, \verb+--+ meaning to do so without modifying
the working area, and \verb+*+ telling it to reset all files.
\verb+git+ will announce that it unstaged changes, and that it un-modified
your modifications to \verb+hello.txt+ using the abbreviated git status
notation. A quick check on the file, however, notes that despite being
unstaged, the file is still unchanged in the working directory. You can check
that out with \verb+status+ again. Often times when you get to the point of
staging things, you would like to keep those changes somewhere. Instead of
discarding them entirely, you can use
}

\begin{verbatim}
git stash
\end{verbatim}

\par{
This will place the changes in a ``stash'' that can be applied later if you'd
like the changes back. Run the command, and you'll see that afterwards the
changes are not applied. You can see all of your stashes via
}

\begin{verbatim}
git stash list
\end{verbatim}

\par{
You can then re-apply the stashes via
}

\begin{verbatim}
git stash apply index
\end{verbatim}

\par{
with index the number of stash (enclosed in curly braces when listed). 
When you apply it, it remains in the list. You can remove the stash via
}

\begin{verbatim}
git stash drop index
\end{verbatim}

\par{
To perform both of the above in the same move for the most recent stash, you
can run
}

\begin{verbatim}
git stash pop
\end{verbatim}

\par{
Stashes are incredibly useful for experimenting with different changes in a
well-recorded and systematic manner; just make sure to clean out stashes as
they become outdated or unnecessary. For now we can just leave it stashed.
}

\par{
If we want to return the file to its previous commit state completely, even in
the working area, we can do a reset with the \verb+--hard+ argument. It is
worth noting that you should be very sure you want to do this, as your changes
will be completely lost. If you create important changes and then use a hard
reset, you will lose them and then be a sinner in the hands of an angry (and
unforgiving) system. If you think there's even a chance at that patch's
usefulness, stash it. Practice a hard reset by applying (not popping) the
stash and then run
}

\begin{verbatim}
git reset --hard
\end{verbatim}

\par{
This puts you back to the previous commit in both your working and staging
area. Note that your stashed changes survive, so you can recover them in the
event of a hard reset.
}

\section{Git History}
\par{
Soon after we've made those changes, a moment of justice occurs. Flanked by the
person who insisted that exclamation point was impermissible, your boss
approaches you and says that exclamation points are in fact okay (because
duh). After savoring some sweet justice and drinking in the apology of the
person who tried to pry away the joy expressed in just a handful of computer
bits, you would like to put the exclamation point back. You \emph{could}
simply change the punctuation back and re-commit it, but that's not a good
expression of what happened. Instead, you should \emph{revert} that change to
make it clear that you're undoing something that should not have been done.
Before we can do that, though, we need to make sure we can access that commit.
There's a simple way to do that. Run
}

\begin{verbatim}
git log
\end{verbatim}

\par{
You should see three commits there. Each one has a \emph{very long}
hexadecimal hash, an author (that's you), an e-mail address, a date, and the
messages associated with those commits. This provides you, at a glance, with
the information you need to find a change, who made it, and how to contact
them, as well as when it was introduced. It cannot be understated how valuable
this is. It will become apparent when you work on software with more complex
bugs how useful it is to track down \emph{when} a problem was introduced and 
\emph{who to ask about it}. Nonetheless, for our purposes this is a lot of
information. It would be nice to see a lot less information: just the summary
message and each commit's hash. We can do that via
}

\begin{verbatim}
git log --pretty=oneline
\end{verbatim}

\par{
This is much cleaner. However, we see that there is this parenthetical that
isn't part of our message: \verb+(HEAD -> master)+. This is telling you what
commit you're comparing your staging and work area to right now (HEAD), and
what branch it's pointing at (master). This will be valuable when we discuss
branching later. This clean version is much easier to parse, but I actually
still think it's a bit much. Personally, I've added an alias called ``slog''
that prints out a shorter (7-character) version of the hash and the message in
red and green, respectively. If you would like to make a detour to use this
macro, paste the following into your terminal:
}

\begin{verbatim}
git config --global alias.slog 'log --pretty="%C(red)%h %C(green)%s"'
\end{verbatim}

\par{
This basically tells git to post the short commit hash in red next to the
short message in green. 

\par{
Now that we can slog through the git history, it's straightforward to find the
commit we'd like to revert. Find it, and copy its short hash (this is easier
on some terminals than others). We can revert it via 
}

\begin{verbatim}
git revert hash
\end{verbatim}

\par{
When you do this, your text editor will launch and prompt you to enter a
commit message. It is already populated with a message indicate that this
commit is a special revert commit, including the old message and hash. You can
add a rationale below the comment block. Go ahead and add something like,
``Reverted due to our boss overriding a silly rule that prohibited ! in text
files.'' Write and exit from your text editor, and a new commit will be
generated reflecting these changes. The commit is reflected when you run
\verb+git log+. Let's say you want to view only the last commit (which is
fairly common) and you want to see all of the details. You can add the
argument \verb+-n+, with \verb+n+ the number of commits you'd like to view in
reverse chronological order. Note that this is compatible with the
\verb+slog+ alias, too, since it's just an alias for the \verb+log+ command. 
}

\par{
Let's say that we want to go back our commit where we've added our wonderful
world message, before all of the ! nonsense, instead of reverting. There can
be cases where this is useful, but I will caution you: this is an easy way of
becoming a \emph{very bad} sinner in the hands of a \emph{very angry} system.
The first way is not so bad: you can do a ``soft'' reset, where you keep the
code in the last commit but set the HEAD (a pointer to where you are in git
history) to a previous commit. You can do this via
}

\begin{verbatim}
git reset --soft hash
\end{verbatim}

\par{
When you do this, however, you will quickly discover why it is so easy to
become a sinner in the hands of an angry system in this way. When you check
the history log, you will see that your previous two commits are gone, and
there are no changes in the file. Why? All you did was make a change and
revert it, so there are no changes to stage. As far as git is concerned, the
state of this snapshot is no different from how it was before the reset. While
it is possible to recover the fact that you made a change and canceled, it is
quite perilous if you're making more complex changes. Fortunately, in this
case it's not so bad. You can run
}

\begin{verbatim}
git reflog
\end{verbatim}

\par{
And you'll see all of the positions that HEAD has been at. You can then reset
to one of those hashes; in this case, you'll want to reset to the hash where
we were at the revert commit. Having done this, you will recover the history
and averted disaster. However, it should be easy to see how such operations
could lead to fairly complicated and irritating situations. As a rule, you
should always commit a subtraction of information or revert a commit instead
of performing a reset. You can avoid catastrophe through the use of branches,
which will be discussed in detail in the next chapter.
}

\par{
While we're poking around, let's go ahead and add another lyric to our file, 
``Is there anybody out there?''
Go ahead and insert that line below ``Hello, world!'' and commit that change.
As soon as we do that, we hear more clomping of feet. It's the boss and our
old nemesis, again full of contrition. You are indeed allowed to add more than
one line of lyrics per song, and so you want to apply that stash that you had
from earlier in the staging section. Because \verb+git+ is superlative at
keeping all of these things in a database, your change is still stashed and
you can see it. Go ahead and pop it, and you will see that the change is
correctly placed in the position below ``What a wonderful world.'' This is
amazing in some sense: we have beaten this poor git log with all sorts of bad
choices in the last few commands, nearly shot ourselves in the foot, and
trusty \verb+git+ has delivered to us a perfect reconstruction of our desired
text file in spite of all of this. Amazing! Go ahead and commit the result.
}

\par{
We take a look at our file, reviewing carefully (as all good programmers
should) and notice that there's something slightly confusing: we have two song
lyrics that are practically on top of each other. Reformat the file so that
each song is labeled and separated by whitespace both before and after that
song, and indent the lyrics. Commit these changes, and view your git log. We
now have 7 different snapshots in time of the file, and an even larger number
of different HEAD positions that have been recorded in ``reflog.'' We have
also used stashes and reversions to reconstruct precisely the file we want,
with a record of all changes that have been made, with a handful of commands.
That's fairly incredible.
}

\par{
For a final set of changes to demonstrate \verb+git+'s flexibility, we're
going to note that at this point there are more song lyrics than hellos in
this file. Let's plan to get rid of the ``Hello, world!'' and move it to a new
file, README.md (every repository should have one!)
Go ahead and first delete that line from \verb+hello.txt+ and the whitespace
after, then commit that change. Now rename the file to \verb+lyrics.txt+ and
run \verb+git status+. We see that the file \verb+hello.txt+ is deleted and
that we now have an untracked file, \verb+lyrics.txt+. This isn't right-- all
we did was rename the file! Fortunately, there's an easy fix. Do a hard reset
via
}

\begin{verbatim}
git reset --hard
\end{verbatim}

\par{
You should now be at the previous commit, having discarded this change that we
don't like (we could have stashed it to be cautious, or used \verb+--soft+ to
keep the changes in the staging area, at least, but we were positive we didn't
want things recorded this way). Fortunately \verb+git+ comes with a facility
to rename things. Just run
}

\begin{verbatim}
git mv hello.txt lyrics.txt
\end{verbatim}

\par{
When you run git status this time, you should see that \verb+git+ now knows
that the file was simply renamed. Go ahead and commit this change-- that's
what we wanted to begin with. Though it seems pedantic to do this, it
dramatically eases workflow if we have a proper record of what we did, as we
saw when we attempted resetting to different commits earlier.
Keeping the log clean with good commits is the easiest way to manage code and
avoid becoming, as you tire of reading, a sinner in the hands of an angry
system.
}

\par{
As a final addition, create a file \verb+README.md+ with ``Hello, world!'' and
commit it. When you've done this, run 
}

\begin{verbatim}
git rebase -i HEAD~9
\end{verbatim}

\par{
You will be presented with a list of all of your short commit hashes and their
messages, with a keyword ``pick'' in front of all of them. There are numerous
things you can do to manipulate the commits listed in the comment block below.
Note that unlike the log you get on the command line, the commits are not in
reverse order. You can now easily amend your history. After a great degree of
soul searching, you have decided that you can erase the part of history where
! was changed to ., and you'd like to amend the first commit message to start
with ``FIRST:''. To do these things, replace \verb+pick+ on line 1 with
\verb+reword+ and replace lines 2 and 3 with \verb+drop+. Even more
sophisticatedly, you'd like to make the file split into lyrics and README to
be conducted in one fell swoop; after all, we broke a rule when we deleted
``Hello, world!'' in one commit and added it back in another: those changes
are mutually dependent. To do that is more convoluted, but we essentially
want to amend our commit message from where we removed ``Hello, world!'' into
something representing the file split, and we want to \verb+squash+ the other
changes into it-- in order words, the commits that happened afterwards will be
merged into a single commit, with an amended message to reflect all of the
conatained changes. To do this, replace \verb+pick+ on line 6 where the
message is removed with \verb+reword+, similar to above, and replace
\verb+pick+ with \verb+squash+ on the two commits below it. After verifying
that this is all correct, save and quit your editor.
}

\par{
Your editor will then launch a new window, asking you to amend the commit
message for the first ever commit. Add ``FIRST'' before the original message,
then save and quit. You'll be brought to the commit where you deleted ``Hello,
world!'' and you should modify the message to represent all of the changes.
Add a body to explain the changes' rationale as well as a summary. When finish
this, the editor will pop up a third time asking you to merge all of the
commit messages. This gives you a chance to verify your message that you
reworded previously. Delete the other messages and move the body of your
reworded commit down to the bottom. After this, you should get a message
saying ``Successfully rebased and updated refs/head/master,'' indicating
success. If something goes wrong, it may be easiest to simply do a hard reset
and try again. Check your files and git log. If there is anything wrong, use
the reflog to try and get back to your previous head, and then try again.
If you ever want to modify just the last commit message, you can just run
}

\begin{verbatim}
git commit --amend
\end{verbatim}

\par{
Now you know how to access and even modify your git history. Just remember to
be cautious, so that you do not become a sinner in the hands of an angry
system.
}

\section{Interactive Modes}

\par{
Sometimes files get big. Really, really big. Then you'll go to make changes,
and there will be changes all over the place that are in disparate locations.
Then you will go to stage them, and realize you're about to commit changes
that do completely different things. Obviously, this violates our rules about
commits, so we should stage and commit them separately. Otherwise, we will be
(say it with me) sinners in the hands of an angry system, because we will have
changes that are difficult to revert. Previously, we used the flag \verb+-i+
with \verb+rebase+. This is an example of an interactive \verb+git+ program.
While it can be useful for amending history, it can also be useful for writing
it to begin with! In your companion repository, there is a file,
\verb+big.txt+. Copy it into this repository; the following command should
work if you followed instructions at setup:
}

\begin{verbatim}
cp ../test_git/big.txt .
\end{verbatim}

\par{
Commit the file, and then take a look at it. Correct all of the lyrics errors,
then run
}

\begin{verbatim}
git add -p big.txt
\end{verbatim}

\par{
This will interactively present ``hunks'' of code to you for you to
selectively stage. First stage only the hunks of ``Row, row, row your boat''
by saying ``yes'' to the first hunk, and ``no'' to the second. Check the git
status to verify that you have some changes staged and some not. Once you've
verified that, commit the first change, noting that the correction is only for
Row, row, row your boat. Checking the workign area and the status should
verify that this was successful. 
}

\par{
You can actually edit to an even more precise level than the ``hunk'' provided
to you by git (which is itself actually based off a tool called ``diff''
provided by GNU}). Re-run the interactive patching program, but when the hunk
comes up, enter \verb+e+ instead for edit. This should launch your editor, and
provide you with instructions for how to edit. Essentially, you want your
changes to reflect how they would look if you ran \verb+git diff+. Therefore,
deleting a line with + equates to removing an insertion that you had put in,
and replacing a - with a space means to leave that line alone. 
Go through and edit the patch
so that only corrections are made for lines with the word spider.
Commit these changes, then commit the remaining corrections separately.
}

\par{
Git offers many interactive facilities similar to this. For a more complete
staging menu, including unstaging and individual file manipulation for many
files, you can use
}

\begin{verbatim}
git add -i
\end{verbatim}

\par{
Follow the instructions similar to interactive staging. Any time a line
appears with a star, simply press enter to confirm that that is the file you
want to interactively manipulate.
}

\par{
These interactive tools are substantially
more precise than GUIs: note that even GitKraken, as of now one of the best
git GUIs, fails to allow users to stage lines at the sub-hunk resolution.
It is no slower to respond y/n to hunks on the command line, so a GUI with
less functionality is at best a pretty memory hog and at worst encourages
users to create bad commits that entangle disparate issues. 
Moreover, many command line tools exist for debugging, compiling, running
scripts, and more, which cannot be done from inside most git clients. In
contrast, command-line git can be rolled into whatever existing command-line
tooling you're using. I will note, however, that on Windows machines, the
interactive git command line is much slower than on Unix machines for reasons
beyond my comprehension. I often wonder if the poor Windows developers
are forced to 
\href{https://en.wikipedia.org/wiki/Eating_your_own_dog_food}{eat their own dog food}
instead of using the high-quality command line tools that Unix users enjoy. 
}
