\epigraph{Whereof we cannot speak we must be silent.}{\textit{Ludwig
Wittgenstein}}

\par{
This is a companion document to the Git Boot Camp. We hope that you'll find it
useful and that it promotes good practices and workflows. Git is an incredibly
powerful tool that can be used on almost any text-based framework, including
documentation or code. If you investigate the test repository that this
document references, you will see that even this document is tracked there,
allowing the authors to carefully manage changes. Though there are many
version control systems that are alternatives to git, git is by far the most
widespread, and few would contend that it is inferior to other systems. Many
projects are on other systems only due to historical reasons, and migrating
repositories can be very challenging on large-scale projects. 
Many projects, including the entire Windows operating system, have
successfully been migrated in spite of this \emph{because} of git's power and
flexibility compared to other systems.
}

\par{
Git is a framework with which assorted changes can be discussed, managed, 
organized, and conceived. Through git, many thousands of people work on the
Linux kernel, making hundreds of thousands of changes a day, and these changes
are funneled into a single version at a moment in time that is testable and
deployable to computers all over the world as the dominant operating system.
These tools underpin workflows of almost every project that uses them. Just as
it is difficult to conceive of lumber without a saw to cut it, it is difficult
to conceive of managing changes without version control. Quickly, the changes
would give way to a chaos of renamed files and folders without any
synchronization between computers. The Wittgenstein quote above is a reminder
that without a framework or langugage (git) to discuss changes, 
it would be impossible to state them coherently. 
Good luck on your git boot camp journey, and we hope
that you will apply the principles inside with enthusiasm and rigor, so that
you might develop better programs more efficiently. 
}
