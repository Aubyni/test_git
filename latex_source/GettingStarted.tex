\epigraph{A journey of a thousand miles begins with a single
step}{\textit{Laozi}}

\section{Core Configuration}
\par{
To begin, we wil configure your very own git profile. Part of the joy of git
is seeing who made the changes we're unhappy with, and so all commit authors
are identified by their settings in their configuration. This is on a per-user
basis, so that a system with multiple users will still allow different
authors. So when we configure, we will use the ``global'' option, but that
represents universality across repositories rather than across the system.
First, let's see what's already in the global configuration.

\begin{verbatim}
cat ~/.gitconfig
\end{verbatim}

should bring up the global git configuration based on your home directory.
This should work inside of windows git bash or any Unix-based system.
You may see things like a filter section, a core section, and a path to a
merge tool. If you haven't set up your name and e-mail already, then they will
not be there. Let's add them with the following commands:

\begin{verbatim}
git config --global user.name "Joshua B. Teves"
git config --global user.email jbtevespro@gmail.com
\end{verbatim}

The first command is \verb+ git +, which tells the
operating system that you'd like to use the \verb+git+
binary. The second option, \verb+config+ 
tells the binary that you're interested in modifying a
configuration of some sort, and \verb+--global+ specifies
that this is the \emph{global} configuration we'd like to edit. 
\verb+user.name+ indicates the section, separated by
a period with the field, that should be filled. By default, name and e-mail
are stored in every commit message. Note that if you're contributing to a
public repository, this means that your e-mail address is publicly visible. 
}

\par{
Now we've told git who we are. It would also be good to set some additional
preferences. You will want to set a command-line-based text editor. I
recommend \verb+nano+ if you don't work on the
command line very much. If you're a command line wizard, then you should learn
\verb+vim+ if you haven't used it already. To set
your text editor preference, use

\begin{verbatim}
git config --global core.editor nano/vim
\end{verbatim}

We won't have a convenient way to test this for a little while, but we can
check to make sure that the changes were written to the config file by using
the previous \verb+cat+ command. Verify that this is the case. 
}

\section{Initializing a repository}
\par{
You will want to have a single folder as the top-level directory of your
repository. I recommend that you use your home directory in most cases. Note
that on Mac, Apple has seen fit to make Home difficult to accesss from Finder,
making it difficult to attach contents in e-mails, etc. For this reason you
may find it irritating to keep repositories in your home on Mac systems; I
don't have a good recommendation for handling that, it's mostly just painful. 
I do not recommend keeping your repositories on a cloud-synced folder such as
Dropbox or Box Sync; it can cause some very strange issues for
synchronization, and hidden files are not handled well. Thus, I recommend
avoiding such a configuration. When you've selected a suitable location,
create a folder called \verb+HelloWorld+ and navigate to it like this:
}

\begin{verbatim}
mkdir HelloWorld && cd HelloWorld
\end{verbatim}

\par{
From there, you will want to initialize a git repository. Note that if you
don't initialize a folder as a git repository, then there is no database for
git to allow you to make changes, understand where changes should go, or
history record to look at and verify. Therefore, almost all non-global git
commands would fail. Initialize your repository via
}

\begin{verbatim}
git init
\end{verbatim}

\par{
You should get a message informing you that the initialization was successful.
If you were to attempt to initialize a git repository as a subfolder of an
existing repository, you would potentially get an error or worse, create a
confusing database enmeshment that is impossible to entangle. For this reason,
you should not put git repositories inside of each other, or you will be a
sinner in the hands of an angry system!
}

\section{Cloning Repositories}

\par{
It is also possible to clone repositories. This is useful when you want to
collaborate with others or work off of existing projects; in other words, the
overwhelming majority of the work that you'll be doing. A clone allows you to
identically replicate the entire history of a project, with all snapshots of
all code states. This is extremely useful for testing, debugging, and
referring to older versions. One advantage of git is that it is a
``distributed'' version control system, which means that each person working
has their own copy of a repository. This allows each person to instantaneously
access snapshots without having to wait for a server to retrieve them. This
will be discussed in much deeper detail later; for now, understand that it is
the nomenclature is correct: you do indeed have a complete clone of a
repository when you run this command. Though you don't yet know how to make
changes in get, for practice, let's go ahead and try to clone a repository.
Be sure to leave your present repository, and then run
}

\begin{verbatim}
git clone git@github.com:jbteves/test_git.git
\end{verbatim}

\par{
Enter the directory, and list the files. You should see the entire contents of
this test repository. In fact, if you were to poke around, you would find this
file (\verb+GettingStarted.tex+)! 
One of the wonderful things about a git repository is that you can
package all related materials together. For now, we won't mess very much with
this repository, but it's good to see how to properly clone a git repository.
Note that if you were to just copy and paste the contents, the git database
would not migrate properly because it's a hidden directory on most Unix
systems. Cloning guarantees that all of the database is transferred correctly
with all history-- this is why straight downloading is not available on sites
like GitHub; you could very easily mangle the code with no recourse.  
}

\par{
Feel free to poke around the repository's components, but the important thing
is that it appears just like any other collection of files. You can view them,
edit them if you like (though I don't recommend that right this second), and
even compile the \LaTeX source. In other words, it's just like any other
folder full of files on the surface. Once you're convinced of this, return to
the repository that you initialized yourself.
}
