\epigraph{entropy: \emph{noun}: lack of order or predictability; gradual
decline into disorder}{\textit{Google Dictionary}}

\par{
Entropy of code is a known problem: over time, an unmanaged project will tend
to disorder that is difficult to reverse. Version Control is the machine that
projects use to fight back against entropy with clear processes. Just like
thermodynamic systems, version control systems also require energy to run.
However, unlike thermodynamic systems, it is typically a lower energetic cost
to run the system than to let entropy take hold. We will embark on a thought
experiment to illustrate why this is the case. 
}

\par{
Imagine that you are a group of just three people working on an application.
Fortunately, this application is written in just one language that all three
people are reasonably familiar with. The goal of the application is to take a
set of data in a known format and generate a "clean" version of that data. The
mathematical principles are straightforward enough, and so it is implemented
quickly. Then one day, a new method is discovered for cleaning the data. The
group agrees that this change should be merged into the existing code.
Suddenly, a horrible discovery is made: there are actually five versions of
the code. There's programmer A's version that has a configuration unique to
their system, programmer B's version which has a version good for one system
and another version for their system at home, and you have a version that
works on your system, plus some extra features that you thought would be neat.
You then all independently attempt to solve different parts of this new method
and integrate them into your codes, and figure that you can resolve the merge
later because you are all very busy people with important things to do, and
this is a simple problem to solve in a mythical future where there is more
time to solve the problem.
}

\par{
A few months later, it is discovered that all three of you are getting
different results on the same data even though the algorihtm is purely
deterministic. You get together and attempt to compare code to find the
discrepancy. To your collective horror, it is impossible to align your source
code because people have different functions in different places, slight
tweaks in the middle of different functions, and no record of previous
versions because they were backed up on some system long ago and overwritten.
By your estimate, there are now at least nine versions of the code despite
there only being three of you because there are intermediate versions of each
person's code which may or may not align with other versions. What a disaster!
Then, a new lab member comes along, picks one of the nine versions, and begins
their own versions. In no time at all, nobody can tell which version is
actually the correct one, and your boss is fuming because nobody can tell them
what state the project is in, because you have to sheepishly ask, ``Which
version?"
}

\par{
I will assert, based on the fact that this is an actual true story that
happened in my undergraduate lab, and something that I have commiserated with
numerous other labs about, the following statement for the scaling of
time spent managing code changes without version control:

\begin{equation}
t_{\text{NOVCS}} = k n^2
\end{equation}

with $t_{NOVCS}$ the time spent and $n$ the number of changes to be made, and
$k$ the time spent for a single change. This
is clearly not ideal. After the above thought experiment, let us consider a
version-controlled alternative.
}

\par{
Programmer A, B, and C are all working on the same code. They discuss who will
work on what pieces, and each has a their own ``branch'' based off of a main
repository that they all have permission to modify. When changes cause
conflicts, it is immediatley evident and they discuss whose change should be
retained. It is a huge headache because programmers A and B don't agree very
often, and C just wants things done quickly, and there is some degree of
bickering. In time, they integrate the new method to their existing repository
and all three have an identical version. Their boss asks them how it works,
and they begrudgingly say it's okay. 
}

\par{
I will assert, again because the above is a real experience, that the
following holds:

\begin{equation}
t_{\text{VCS}} = K n
\end{equation}

with $K > k$ the effort per change, and $n$ again the number of changes.
Essentially, each change costs more time, but there are fewer changes to
manage. This is the foundational argument that you bank on with a version
control system: even though $K > k$, $n^2 >> n$, so that in the long run,

\begin{equation}
t_{\text{VCS}} << t_{\text{NOVCS}}
\end{equation}

One could argue that $n$ may be small for small projects, but practically
speaking, the number of possible states for a program is large, therefore the
number of possible changes is also large. 
}

\par{
I would contend that the first thought experiment is an example of what I call
``sinners in the hands of an angry system'': they have sinned by not
organizing their code, and the system is angry because they cannot
concentrate their code into one version. It is possible also with version
control to become a sinner in the hands of an angry system, and I will point
this out where I think that is a possibility. Mostly this happens when people
stray from best practices, which I try to outline very clearly.
}

\par{
Now that you've been convinced of the need for version control, let's discuss
more precisely what it is. First, we'll define some terms:
}

\par{
A \emph{patch} is a set of changes. Patches are involved at all stages of the
version control process, and must be carefully managed and designed. 
}
\par{
A \emph{branch} is a set of patches which, taken together, represent a large
work unit. To explain the difference, consider implementing a calculator. A
patch might be a new function to perform addition, but the branch which adds
addition functionality might also include patches for parsing the + sign,
handling errors, and even displaying the result properly. 
}

\par{
Sometimes patches and branches can be much smaller: a patch might fix a bug, and a branch may
simply represent that bugfix. Other times, a branch might be much larger than
a single feature. On most systems, the branch which contains the current
deployed version of an application is called \emph{master}. 

\par{
This is typically
a heavily protected branch, with people waxing poetic on who should and should
not have the ability to write to it, policies in place to avoid malicious
attacks on it or even incompetent additions to it, and so on. Typically in
open source projects, \emph{review} is the process by which a patch or branch
is commented on by the project maintainers to ensure it is ready to add to the
master branch. It is best practice to perform review for all changes.
}

\par{
In a VCS, you typically want more than the ability to just add patches. It is
useful to have tools to compare patches, manage the timing of access to the
master branch so that patches are applied in the correct order, revert patches
that are considered ``bad,'' and so on. Git offers more tools than any other
VCS that I am aware of, and these tools generally work extremely well. I will
briefly go over the main features of git conceptually, before going into the
next chapter, where you will get your feet wet.
}

\par{
Git is, at its heart, a database with near-instant access time to every entry.
Each entry is stored as a \emph{hash}-- a globally unique identifier to a
snapshot in time of the repository. Information is stored as hidden files and
directories within the repository itself, which are then transferred to any
clones of the repository. In git, you can create a new entry into its database
by creating a branch or with a \emph{commit}-- a set of patches that you would
like to be associated with a branch. Because access to any snapshot is
effectively instantaneous, you can very quickly move between many different
versions of the code. It is possible, in other words, to move instantaneously
between any arbitrarily separated code states. You can switch easily between
fixing a bug and adding a feature, between your version and someone else's
version, and between stable and unstable code, without an appreciable cost in
time or memory. For reference, git was designed to allow random access to any
commit in under one second, specifically for the Linux kernel (roughly 15
million lines of code!)
}

\par{
With this flexibility at your fingertips, your ability to maneuver through
changes is dramatically increased. You will discover that it is easy for two
programmers to work on deeply intertwined pieces of code and immediately see
the changes manifest and conflict. You can have multiple independent
development projects occur in parallel and join them seamlessly in seconds,
which uncover accidental conflicts just as quickly. When things must be
reverted, it is done so in a non-destructive way because the database
reference is updated with the reversion. You can tag important versions of the
code, leave detailed messages explaining changes, and even rewrite history
non-destructively (albeit carefully). 
}

\par{
In summary, version control allows people to fight back against the entropy of
code, consolidating all changes into one consensus-driven version. Because git
is essentially a set of snapshots with near-instantaneous access, you can use
it to instantly change between code versions. This allows you to, in practice,
work on any set of problems with any number of people and instantly see
conflicts, merging changes instantly where there are none. So let's get to it!
}
